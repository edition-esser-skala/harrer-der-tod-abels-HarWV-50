\documentclass[tocstyle=ref-genre]{ees}

\begin{document}

\eesTitlePage

\eesCriticalReport{
  –    & –       & –     & In \A1, lyrics are generally illegible.
                           Thus, they have been adopted from \A2. \\
       & –       & org   & In \A1, bass figures only appear in the following
                           movements and bars: 1.2 (110), 1.4 (28, 4th \quarterNote; and 32, 2nd \halfNote), 1.7 (147). All remaining bass figures have been added by the editor. \\
  1.1  & 23f     & vl 2, vla & bars empty in \A1 \\
  1.2  & –       & –     & Grace notes, trills, and slurs in the rhythms
                           \eighthNoteDotted-\thirtysecondNote–%
                           \thirtysecondNote and \sixteenthNoteDotted–%
                           \sixtyfourthnote–\sixtyfourthnote
                           have been tacitly emended.
                           Grace notes have been tacitly added in bars
                           16 (fag 2), 17 (ob grande 2), 20 (fl 1),
                           34 (fl, ob grande), 54 (fl), 58 (fag 1),
                           74 (fl 2, ob grande, vl 2), 75 (fl, vl),
                           131 (fag 1), 146 (ob grande 1), 173 (fl),
                           and 181 (fl 2, ob grande, vl 2, vla). \\
       & 5       & fl 2  & in \A1 unison with fl 1 \\
       & 7       & vl 1  & 2nd/3rd \eighthNote\ in \A1: g′4 \\
       & 7       & vl 2  & 2nd/3rd \eighthNote\ in \A1: e′4 \\
       & 11      & fl 2, vl 2 & 2nd \eighthNote\ in \A1: c″8 \\
       & 83      & fl 2  & 1st \eighthNote\ in \A1: g″8 \\
       & 119–121 & fl    & bars missing in \A1 \\
       & 123     & fl 1  & 1st \quarterNote\ in \A1: c′′′4 \\
       & 130     & ob grande 2 & 2nd \eighthNote\ in \A1: g′8 \\
       & 130     & fag 2 & 2nd \eighthNote\ in \A1: c8 \\
       & 171     & vla   & 1st \quarterNote\ in \A1: e′4 \\
  1.6  & 60–65   & ob    & full measure rests in \A1 \\
       & 86      & vl 1  & grace added by editor \\
  1.7  & 13      & vl, vla & 2nd \halfNote\ in \A1: \crotchetRest–g′4 \\
       & 61      & soli  & 1st \quarterNote\ in \A1: a′8–f′8 \\
  1.8  & –       & ob    & 12
       & 2       & vla   & 4th \eighthNote\ in \A1: \sharp g′8 \\
       & 24      & vla   & 1st \eighthNote\ in \A1: a8 \\
       & 30      & soli  & grace note added by editor \\
       & 32      & vl 1  & 3rd \quarterNote\ in \A1: e″8–d″8 \\
       & 32      & vl 2  & 3rd \quarterNote\ in \A1: \sharp c″8–h′8 \\
       & 42      & vl 2  & 1st \quarterNoteDotted\ in \B1:
                           \sharp g′4–\quaverRest \\
       & 46      & soli  & grace note added by editor \\
       & 55      & vl    & grace note added by editor \\
       & 65      & soli  & bar in \A1: \wholeNoteRest \\
}

\eesToc{
\begin{movement}{intro}
\end{movement}

\part{prima}

\begin{movement}{wunderbarer}
  \voice[Abel]
  O wunderbahrer Gott! an allen deinen Werken
  kan man die Allmacht merken.
  Ich will mein Leben lang [?]
  von deinem großen Nahmen ſingen
  und täglich ein Gelübde dir
  mit Ehrfurcht und viel Dank
  statt der Bezahlung bringen.
  Ihr Völker, preiſet ihn mit mir,
  laß’t ſeinen Ruhm erſchallen.
  Gott [\A2: Ihm] hat mein Opffer wohlgefallen.
  Ach, wer bin ich,
  mein [\A2: o] Schöpfer, daß du dich
  mein ſogar liebreich angenommen
  und ein elendes Menſchenkind
  ſo werthgeacht, daß es ganz unverdient
  zu ſolcher Gnade kommen.

  \voice[Cain]
  Warum iſt Abel ſo vergnügt?
  Was iſt ihm denn ſo gutes wiederfahren,
  daß Luſt und Wonne ſich in ſeinen Augen paaren?

  \voice[Abel]
  Ach theile doch mit mir,
  geliebter Cain, meine Freude,
  ach freuen wir uns beyde,
  weil ich mit Wahrheit ſagen kan,
  Gott ſah’ mein Opffer gnädig an.

  \voice[Cain]
  Und Abel unterſteht ſich das zu ſagen,
  du ſchmeichelſt dir zu viel.

  \voice[Abel]
  Wie ſoll ich etwan Zweiffel tragen?
  Umſonſt läßt Gott nicht was beſonders ſehn,
  wie dazumahl geſchehn.
  Denn als ich von den Erſtlingen der Heerde
  die feſten und die beſten nahm
  und Ihm, von dem doch alles kam,
  mit Danken und Gebeth ein Opffer brachte,
  ward ich etwas gewahr, das mich erſtaunend machte.
  Ich hatt’ es kaum auff den Altar gelegt,
  ward auch ſogleich [\A2: ſo ward auch gleich] des Himmels Kopff bewegt,
  ein Feuer fuhr herab gleich einem hellen Blitze,
  und durch deßalben Brand und Hitze
  ward alles in ſo kurzer Zeit verzehrt,
  als kaum ein ſchneller Blick von unſern Augen währt.

  \voice[Cain]
  Das iſt ja wunderſam! allein, es kan nicht fehlen,
  ſonſt würdest du mirs nicht erzehlen.
  Ich habe gleichfalls Gott von meiner Heerde Frucht
  ein Opffer zugedacht [\A2: ausgeſucht]
  und ihm wie du gebracht,
  doch ſowas ſeltens nicht erblicket.
  Und da es ſich ſo ſchicket,
  daß Eva kommt, ſo ſoll ſie es auch wißen,
  damit das ganze Hauß ein Feſt wird halten müßen.
  Komm, Eva, höre mich,
  ich habe dir was vorzubringen.
  Es redet Abel izt von lauter Wunderdingen,
  die Gott an ihm gethan.
  Nimms aber auch vor eine Wahrheit an.

  \voice[Eva]
  Du zweifelſt wohl vielleicht, ich hab’ es ſelbſt geſehen.

  \voice[Cain]
  Und was?

  \voice[Eva]
  Ein Feuer aus den Wolken gehen,
  dadurch das Opffer gleich verzehret ward.

  \voice[Cain]
  Iſt das gewiß?

  \voice[Eva]
  Was meine Augen ſchauen,
  drauff kanſt du ſicher bauen.

  \voice[Cain]
  O Wahrheit, die vor mich nicht ſchrecklicher kan ſeyn.

  \voice[Eva]
  Ach Kinder, bildet euch darauff nichts ein,
  laß’t euch den Stolz nicht ſchmeichelnd nähren,
  laß’t euch die Mißgunſt nicht verführen.
  Denn jeder that wie er jeds wahl,
  den Dienst und Pflicht,
  den er Gott ſchuldig war. Vergeßet aus Gehorſam nicht,
  was Adam euch befahl;
  du Cain geh’ ins Feld, und Abel zu der Heerde,
  daß euer Thun verrichtet werde.
  Greifft alles Werk in Gottes Nahmen an,
  ſo iſt auch alles wohlgethan.
  Wenn wir mit Ihm die Hand zu Werke legen,
  ſo liebt er uns und ſchenkt uns ſeinen Seegen.

  \voice[Abel]
  Nichts angenehmers kan nur wohl geſchehn,
  als wenn man mich heißt zu der Heerde gehn.
  [\A2: als wenn ich ſoll zur Heerde gehn.]
  Sie iſt das einzige, an dem ich mich auch labe,
  ob ich dabey gleich Müh und Arbeit habe.
\end{movement}

\begin{movement}{ichbinein}
  \voice[Abel]
  Ich bin ein Hirte, der ſein Leben
  vor ſeine Schaaffe darzugeben
  aus Liebe kein Bedenken trägt.
  Sie hören meine Hirtenlieder,
  ich kenne ſie, und ſie auch wieder
  den Schäfer, der ſie zärtlich pflegt.
\end{movement}

\begin{movement}{wasfehlet}
  \voice[Eva]
  Was fehlet, Cain, dir?
  Du ſiehſt ja [\A2: Wie ſiehſt du] ſo zerſtöhrt in deinen Augen aus.
  Entdeck es mir
  und ſag es frey heraus.
  Du redeſt nichts und ſchlägſt die Augenlieder
  voll Unmuth und Verdruß zur Erde nieder.

  \voice[Cain]
  Es geht mir nicht darnach, daß ich kan fröhlich ſeyn.

  \voice[Eva]
  Wie [\A2: Ey] ſolle dich des Bruders Ehre nicht erfreun!

  \voice[Cain]
  Die eben iſts, die mich empfindlich kränket.

  \voice[Eva]
  Iſts möglich, daß dein Herz ſo übel denket,
  ſo läßt du dich denn des verdrießen,
  was wir vor Gottes Gnade ſchützen müßen?
  Dies iſt des Schöpffers Art, der iſt ein Schadenfroh.
  Mein Cain, haader doch nicht ſo.
  Wir müßen unſern Nächſten ja was Gutes gönnen,
  weil wir dergleichen allewohl auch hoffen können.
  Reiß dieſes Unkraut aus, eh es auf Wurzel faßet.
  Wenn Cain ſeinen Bruder haßet,
  ſo wird ſich täglich etwas andres finden,
  worauff er ſeinen Haß kan gründen.
  Mein Sohn, ach thu mir doch ſolch Herzeleid nicht an,
  daraus noch mit der Zeit was ärgers kommen kan.
\end{movement}

\begin{movement}{einwasser}
  \voice[Eva]
  Ein Waßer, das aus trüben Quellen
  pflegt ſeine Fluthen herzuſtellen,
  laufft niemahls klar ins Meer hinein.
  Wie will es alſo hier auff Erden
  nah mit euch beyden Brüdern werden,
  wenn dieſes ſoll der Anfang seyn?
\end{movement}

\begin{movement}{sosollder}
  \voice[Cain]
  So ſoll der junge Bruder denn
  vielmehr als ich der Ältre ſeyn,
  und ich dabey ganz unempfindlich bleiben,
  daß heißet die Geduld auffs allerhöchſte treiben.
  Er hat die Ehre nur allein,
  ich aber nichts als Schimpff und Schande,
  und niemand klaget mich in ſo betrübtem Stande.
  Ja fang ichs noch ſo liſtig an
  und will ihn nur bedrücken,
  muß ſich doch etwas ſchicken,
  daß ich auch an ihn kommen kan,
  dadurch wird er von Zeit zu Zeiten größer,
  allein mit mir wirds niemahls beßer.
  Es wächſt ſein Übermuth, in den ich mit Bedruß,
  der kaum zu dulden ist, doch alles leiden muß.
\end{movement}

\begin{movement}{ichlebenur}
  \voice[Cain]
  Ich lebe nur ſelber zur Marter und leide,
  indem ich den Wohlſtand des Abels beneide.
  Ich raſe bey tauſend verdrießlichen Quälen,
  die Mißgunſt verzehret [\A2: verzehrt mir] die Kräffte der Seelen,
  er iſt mir zuwider, ich weiß nicht warum.
  Ja will ich die Urſach des Haßes ergründen,
  ſo ſuch ich ſie ſtündlich und kan ſie nicht finden,
  indeßen ſo wächſet [\A2: mehrt ſich] der Eifer im Herzen,
  und wollt ich, und könnt ich gleich alles verſchmertzen,
  ſo iſt es vergebens, und lohnt mir nicht drum.
\end{movement}

\begin{movement}{warumergrimmest}
  \voice[Die Stimme Gottes]
  Warum ergrimmeſt du?
  Warum verſtellſt du die Gebenden
  und willſt nicht bey [\A2: in] dir richtig wenden?
  Iſts nicht alſo? wenn du dich fromm erziehſt,
  ſo biſt du angenehm, hingegen ſiehe zu,
  wo du dein Herz zur Boſheit neigſt,
  ſo ruht die Sünde vor der Thüre.
  Nimm dich in Acht, daß ſie dich nicht verführe.
  Wer ihr den Willen bricht,
  an dem gewinnt ſie keine Herrſchafft nicht.

  \voice[Cain]
  Es iſt noch nicht genug, daß Abels Ehre
  mich in dem innerſten der Seele plagt,
  nun kommt es gar ſo weit, daß ich Verweiſe höre,
  die man mir ins Geſichte ſagt.
  Allein, es kommt die mir verhaßte Seele
  dort mit der Heerde raus.
  Wie ſieht er voll Vergnügen aus,
  wie viel weiß er ſich einzubilden,
  wie wird er nicht auff den Gefilden
  recht trotzig thun. Ich will ihm aus dem Wege gehn.
  Er ist vor mir nicht auszuſtehn,
  ich kan die Augen, die [\A2: ſo] vor Eifer brennen,
  ihm kaum vergönnen,
  und alles, was er thut, das ärgert mich.

  \voice[Abel]
  Mein Bruder – wie? entfernſt du dich?

  \voice[Cain]
  Ein ſchlechter Menſch muß ſich zu ſeinesgleichen finden,
  bey dir durfft ich es mir nicht unterwinden,
  weil dich der Himmel ſelbſt zu großen Herren zählt
  und dich als ſeinen Liebling auserwählt.

  \voice[Abel]
  Wie redet Cain doch ſo ungewöhnlich.

  \voice[Cain]
  Du kommſt vielleicht darum geſöhnlich,
  daß ich dir Glücke wünſchen kan,
  weil Gott ſchon wiederum an dir
  was großes hat gethan.

  \voice[Abel]
  Ich rühme mich ja nichts als Gottes Gnadengaben,
  die wir von ihm und niemand anders [\A2: ſonst von niemand] haben.

  \voice[Cain]
  Du biſt es nur allein, der, wenn man Opffer bringt,
  den Himmel zur Erhörung zwingt.
  Zu ſolcher Gnade werd ich wohl nicht kommen,
  denn du haſt ſie vor dich ſchon weggenommen.

  \voice[Abel]
  Was hör ich doch von dir,
  ach wie verſündigeſt du dich an mir.
  Ich kan ja nicht davor,
  daß Gott dich [\A2: dich Gott] nicht gnädig angeſehn.
  Vielleicht iſt es [\A2: Es iſt vielleicht] darum geſchehn,
  daß er dir [\A2: dadurch] weiſen wollen,
  wie du noch frömmer werden ſollen.
  Ach Herr, wie wird dein Thun ſogar ſehr ſchlecht geacht,
  du weiſeſt jedermann die rechten Wege,
  und manche weichen ab auff Nebenſtege,
  in denen er ſich dann verkehrte Meynung macht!
  So fangt die Schlange Gifft aus den geſündſten Kräutern,
  woraus [\A2: daraus] die Bienen ſonſt den beſten Honig läutern.

  \voice[Cain]
  Verwegner Menſch, halt ein,
  willſt du ſogar vermeßen ſeyn,
  daß ich von dir noch ſoll Verweiſe hören,
  ich glaub, ich ſoll dich gar als meinen Meiſter ehren?

  \voice[Abel]
  Ich red ja mit dir nicht als wie ein Feind,
  was ich geſagt, iſt herzlich gut gemeynt
  und kommt aus brüderlicher Liebe.

  \voice[Cain]
  Weg, weg mit deiner Bruderliebe.

  \voice[Abel]
  Und alſo ſoll der Haß –

  \voice[Cain]
  Ja, den verdieneſt du!

  \voice[Abel]
  Warum?

  \voice[Cain]
  Schweig, geh, laß mich in Ruh,
  ich kan die Rede kaum erdulden [\A2: mehr dulden].

  \voice[Abel]
  Ach ſage mir doch mein Verſchulden,
  ich habe dir ja nichts gethan.

  \voice[Cain]
  Diß eben iſts, was ich nicht leiden kan.

  \voice[Adam]
  Was habt ihr, meine Kinder,
  vor einen harten Streit?
  Ach brauchet doch Beſcheidenheit
  und ſeid gelinder.
  Stirbt ſchon die Brüdertreu,
  da kaum die Welt entſtanden,
  ſo glaub ich, daß ſie gar nicht mehr vorhanden
  und irgends wo zu finden ſeyn.
  Ihr ſeyd ein Fleiſch und Blut, aus einer Mutter Leibe,
  von einem Mann und Weibe,
  und doch von ſo verſchiednem Sinn.
  Wo will es endlich hin,
  was wird auff dieſer Erde,
  die Gott um unſre Mißethat
  ohndem verfluchet hat,
  noch mit dem Menſchen werden?

  \voice[Cain]
  Ach Vater, ſtraff nur auch des Abels Übermuth
  und gieb mir nicht allein den Verruff anzuhören,
  der mir zu viel, und ihm zu wenig thut.
  Er iſts, der will den Frieden ſtöhren,
  er, er iſt Schuld daran,
  daß ich mich nicht kan faßen,
  gerechten Eifer auszulaßen.
  Warum? Er ſieht mich nur verächtlich an,
  denn ſeit er glaubt, er ſteh bey Gott in hohen Orden,
  ſo iſt ſein Stolz ganz unerträglich worden.

  \voice[Adam]
  Ich glaubte dir gewiß,
  allein ich kenn euch beyde,
  drum alles dieß, was du geſagt,
  kömmt nur aus lauter Neide,
  und eine Raſerey verblendet dich.
  Es ſcheint dir Abel ärgerlich,
  dieweil er mir von Tugend wegen weichet
  und dir in keinem Stücke gleichet.
  Du biſt ihm darum fremd,
  weil er es redlich mit dir meynt.
  Thu es ihm vielmehr nach, ſey ihm nicht ſo zuwieder,
  und lebt doch wie die Brüder.
  Mir kommt ein Grauen an,
  weil ich den Ausgang ſchon zum Voraus ſehen kan,
  die Sünde bringet dich zu ſolchen böſen Thaten,
  bedenke, was du thuſt, und laß dir rathen.
\end{movement}

\begin{movement}{daslicht}
  \voice[Adam]
  Das Licht ſcheint dir noch eine Weile,
  bedenk es doch, mein Sohn, und eile,
  entweich aus dieſer Finſterniß.
  Wer in denſelben Schatten ſtehet,
  der weiß nicht, wie und wo er gehet,
  und iſt des Weges nie gewiß.
\end{movement}

\begin{movement}{}
  \voice[Cain]
  So hat denn Abel nun bey allem Recht,
  und Cain iſt allein der Sünde Knecht.
  Ich glaube, daß auch der noch auff der Erde
  gebohren ſollen werd,
  in Mutterliebe nur zu werden ſeyn.
  Hier kommt die Mutter auch, ich bilde mir ſchon ein,
  hab ich die ganze Welt zu Feinden,
  ſo iſt ſie ebenfalls auch nicht an meinen Freuden.

  \voice[Eva]
  Mein Sohn, was ſageſt du?
  Du biſt dein eigner Feind und ſtöhreſt deine Ruh.

  \voice[Adam]
  Es hilfft bey ihm kein Reden und kein Sagen,
  drum keine Lehre kan er mehr ertragen.

  \voice[Eva]
  Noch glaub ichs nicht an ihm! Ich habe das Vertrauen,
  wir werden ſeine Beßrung ſchauen.
  Ich ſeh es ſchon, die Reue kommt ihn an,
  wie du und ich nach unſerm Fall gethan.
  So geh denn hin, mein lieber Sohn,
  Erweiſe, daß mein zärtlich Hoffen
  bey [\A2: an] dir vollkommen eingetroffen.
  Sprich deinem Grimm und Zorne völlig Hohn,
  laß't uns durch brüderlich Umarmen wißen,
  daß unter auch ſich Lieb und Freude küßen.

  \voice[Abel]
  Ich bin dazu bereit.

  \voice[Cain]
  Ach das geſchiehet nicht in Ewigkeit.

  \voice[Eva]
  Wie! was? entferneſt du dich gar,
  nun wird dein Groll nicht offenbahr,
  ſo [\A2: ach] geh doch hin und zeig die Beßerung des Lebens.

  \voice[Cain]
  Schweig, Eva, ſchweig, es [\A2: dein Reden] iſt vergebens.

  \voice[Eva]
  Soll meine Müh vergebens ſeyn!
  Drückt dir mein Weinen nicht Empfindung ein,
  ſoll denn ein Fleiſch und Blut,
  ſo ich in meinem Schoß getragen,
  ſich mit niemand ſo entzweyen und verklagen!
  Mein Sohn, ach faß dir einen Muth,
  geh und benimm du dich, laß mich ſolch Wiederſtreben
  und dieſen Jammer nicht erleben.
  Wie daß dein harter Sinn die Mutter wiederbringt,
  die dich als ein klein Kind an ihrer Bruſt geſäugt,
  die dich mit Weh und Schmerz gebohren,
  an der des Höchſten Fluch
  durch dich als Erstgeburth ſich offenbahrt,
  die Leyden dadurch überführet ward,
  daß ſie ſein Ebenbild verlohren.
  Vertrage dich mit ihm, vergiß doch den Verdruß.

  \voice[Cain]
  Weil du es ſo befiehlst, ſo ſeh ich daß ich muß.

  \voice[Eva]
  Wie glücklich bin ich nun, wie glücklich meine Zähren,
  die Fried und Ruh geſtifft.
  Nunmehr ruht aller Feindſchafft Gifft.

  \voice[Adam]
  Ach, ach, ich glaube nur, es wird nicht lange währen!

  \voice[Eva]
  Warum?

  \voice[Adam]
  Man weiß nicht was geſchieht,
  denn wie das Meer nicht immer ruhig bleibet,
  und durch die Wellen Koth und Unflath vor ſich treibet,
  ſo hat, wer gottloß iſt, auch keine Feinde nicht.
  Er ſcheint die Stille zwar wohl äußerlich zu haben,
  allein im Herzen liegt die ärgſte Stimm vergraben.
\end{movement}

\begin{movement}{}
  \voice[Chor]
  Verruchter Haß und Neid, du Kind der Eigenliebe,
  du Wurzel aller Laſter Triebe,
  der ſelber ſich ſo ſchädlich nicht
  als wie der Roſt den Stahl verzehrt.
  Du biſt dem Unkraut gleich, das alles wiederdrückt
  und auch die beſte Saat erſtickt.
  Ach Herr, bewahr uns doch vor dieſer großen Sünde,
  gieb, daß ſich unſer Herz darum in Lieb’ [\A2: die Liebe ſich darum in uns] entzünde.
  Weil du die Liebe ſelber biſt,
  ſo lebet der in dir, der ihr beflißen iſt.
\end{movement}

\part{seconda}

\begin{movement}{}
  \voice[Cain]
  Ich habe mich entſchloßen:
  Nun ruh ich eher nicht,
  als biß ich Abels Blut vergoßen.
  Was iſt den wohl mit ſolcher Freundſchafft ausſpricht,
  die mir von Herzen geht, die mir nur Unluſt bringet,
  und die aus aller meiner Pein
  die gröbſten Fehler zwinget.
  Er kommt. Mein Herze kalt, zwar lauter Rache,
  allein,
  ein freundliches Gefühl ſoll die Verſüßung ſeyn,
  dadurch ich ihn recht ſicher mache.
  Geliebter Bruder komm, ich warte dein.

  \voice[Abel]
  Wie angehm klingt das! wie lieblich und wie fein
  ist es nicht anzuſehn, wenn Brüder friedlich leben
  und durch der Freundſchafft [\A2: Eintracht] Band einander ſich ergeben.

  \voice[Cain]
  In dir ergeb ich mich, mein Bruder, ganz und gar,
  ich bin nicht mehr der Menſch, der ich vor dieſen war.
  Vergiß nur du, was vorgegangen.
  Es ſoll nunmehr dahin gelangen,
  daß Cain allen Groll vergißt,
  daß unter uns ſich wahre Freundſchafft küßt.
  Komm, geh’ mit mir.

  \voice[Abel]
  Wohin?

  \voice[Cain]
  Aufs Feld, von hier nicht ferne.

  \voice[Abel]
  Ach ja, das thu ich herzlich gerne.
  Was aber ſoll denn die Verrichtung ſeyn?

  \voice[Cain]
  Es fällt mir ein,
  ich will durch Buße Gott verſöhnen,
  und ſeine Straffe abzulehnen
  ein Opffer thun.

  \voice[Abel]
  Allein, das Schlachtſchaaf fehlet dir.

  \voice[Cain]
  Ach nein, es iſt ſchon hier.

  \voice[Abel]
  Man muß bey ſolchen Zeiten
  ſein Herz auch wohl bereiten.

  \voice[Cain]
  Diß alles iſt geſchehn.

  \voice[Abel]
  So wird dein Opffer Gott [\A2: auch Gott dein Opffer] wohlgefallen.

  \voice[Cain]
  Ach ja [\A2: Gewiß], er liebet es vor allem,
  das man ihm bringen kan. Komm nur und laß uns gehn.

  \voice[Eva]
  Wo eilt ihr Kinder hin!

  \voice[Abel \& Cain]
  Auffs Feld.

  \voice[Eva]
  Ach was vor Freude
  ſeh ich an mich, ſo lebt ihr beyde
  wie Brüder zugehört.

  \voice[Cain]
  Komm, Abel, laß uns eilen.

  \voice[Abel]
  Ich komme gleich, den Augenblick.

  \voice[Cain]
  Du ſagſt, du kömmſt, und bleibeſt doch zurück.
  Was nutzet dein Verweilen?

  \voice[Abel]
  Geliebte Mutter, lebe wohl.

  \voice[Cain]
  Komm, komm, der Tag verſtreicht.

  \voice[Eva]
  Wie kommts, daß Abel izt ſo ſchmerzlich vor mir weicht?

  \voice[Abel]
  Ich weiß es nicht warum.

  \voice[Eva]
  Du bleibeſt seuffzend ſtehen,
  als ſollt ich dich nicht wiederſehen.
  So rede doch, mein Kind, antworte meinen Fragen.
  Was will dergleichen Angſt, die du bezeugeſt, sagen?
\end{movement}

\begin{movement}{}
  \voice[Abel]
  Indem ich nun muß von dir ſcheiden,
  ſo ahnt mir was durch ängſtlich leiden,
  was/das ich nicht ſelber ſagen kan.
  Ich geh’ von dir mit ſchwerem Herzen,
  ich ſehe dich mit lauter Schmerzen
  und jammervollen Augen an.
\end{movement}

\begin{movement}{}
  \voice[Abel]
  So lebe denn, geliebte Mutter, wohl.

  \voice[Eva]
  Nun geht er voller Wehmuth fort
  und läßet mich betrübt an dieſem Ort.
  Wem ſo ein Abſchied nicht beweget,
  der muß von Stahl und Eiſen ſeyn.

  \voice[Adam]
  Was nimmt dich, Eva, vor Betrübniß ein?
  Dein Auge thränt, wer hat denn diß erreget?
  Du denkeſt wirklich,
  es ſey nur ein verſtelltes Weſen,
  was Cain läßt aus ſeinen Augen leſen,
  und haſt der Wahrheit Grund verricht?

  \voice[Eva]
  Ich weiß von keinem Leide,
  vielmehr empfind ich Freude,
  daß ſie nun wieder Freunde ſind,
  und unter ihnen ſich kein Zanken mehr entſpinnt.

  \voice[Adam]
  So ſaget Eva wohl, allein, ich ſollte meynen,
  zur Freude ſchicke ſich nicht ſo ein ängſtlich Weinen.
\end{movement}

\begin{movement}{}
  \voice[Adam]
  Wenn betrübte Zähren fließen,
  kan man wohl mit Wahrheit ſchließen
  daß ein innerliches Quälen
  ſie aus unſern Augen gießt.
  Soll man das vor Luſt erkennen
  und vollkommen Freude nennen,
  wenn dir Wehmuth einer Seelen
  ſich im Antlitz finden [\A2: ſehen] läß’t?
\end{movement}

\begin{movement}{}
  \voice[Eva]
  Mein Adam mein,
  ich habe Urſach froh zu ſeyn,
  drum, daß mein Auge weinet,
  geſchiehet nicht aus Leid,
  es iſt die Wirkung einer Zärtlichkeit,
  die meine Bruſt in ſich verſpühret.
  Des Abels Reden haben mich ſo ſehr gerühret,
  als er mit ſeinem Bruder kam
  und ſo beweglich Abſchied nahm.
  Dir hätt’ es auch das Herz genommen,
  wenn du ſie ſelbſt geſehn
  ſo friedſam miteinander kommen [\A2: gehn, ſo liebreich kommen].

  \voice[Adam]
  Und wohin wollen ſie denn gehn?

  \voice[Eva]
  Auffs Feld.

  \voice[Adam]
  Ach Gott, wenn Cain nur die Tücke nicht verſtellt,
  die große Freundſchaft ſcheint mir ſehr verdächtig.

  \voice[Eva]
  Er iſt kein Tyger nicht und ſeiner Sinnen mächtig.

  \voice[Adam]
  Wer Laſter an ſich hat [\A2: hegt], iſt ärger als ein Vieh,
  denn die Vernunfft verdunkeln ſie.

  \voice[Eva]
  Wie kan doch der Verdacht auch aus den beſten Sachen
  die ärgſten Schlüße machen,
  und ſieheſt alles vor gefährlich an,
  diß kommet durch den Fall, den ich und du gethan.

  \voice[Adam]
  Ich weiß mehr als zuwohl, daß die Gedanken trügen,
  doch dieſes kan die Furcht nicht überwiegen,
  die mich im Herzen nagt,
  ſie macht mich dergeſtalt verzagt
  daß ich ohn weiteres Verweylen
  gleich auff den Füßen mach, muß ſie zu finden eilen.

  \voice[Eva]
  Es iſt mehr als zu wahr, ſeitdem wir durch den Fall
  das Elend bau’n und uns mit Kummer nähren,
  ſo ſuchen wir die Ruh faſt überall,
  und niemand außer Gott kan ſie gewähren.
  Doch dort kommt Cain her und ganz allein.
  Wo muß der Bruder Abel ſeyn?
  Er ſieht ſo ſchüchtern aus, als hätte ſein Gewißen
  die größte Mißethat zu büßen.
  Wohin, mein Sohn? Warum entfliehſt du mir?
  Was fehlet dir?
  Was hat dich ſo [\A2: denn] erſchrecket?

  \voice[Cain]
  Welch Unfall führet gleich die Mutter her.

  \voice[Eva]
  Ach Gott! du biſt ja überall mit Blut beflecket,
  als hätteſt du jemand todt geſchlagen.

  \voice[Cain]
  Nun muß ich fort, ſie möchte mich
  auch um den Abel fragen.

  \voice[Die Stimme Gottes]
  Wohin, du Böſewicht?
  Wo iſt dein Bruder hingekommen?

  \voice[Cain]
  Soll ich des Bruders Hüter ſeyn? Ich weiß ihn nicht.

  \voice[Die Stimme Gottes]
  Was haſt du vorgenommen?
  Es ſchreyet deines Bruders Blut
  zu mir um Rache von der Erden.
  Auff dieſer ſollſt du nun von mir verfluchet werden.
  Weil ſie ihr Maul hat auffgethan
  und nahm des Bruders Blut von deinen Händen an.
  Der Acker, den du bauſt,
  ſoll dir hinfort nicht ſein Vermögen geben,
  unſtätt und flüchtig ſollſt du leben.

  \voice[Cain]
  Ach Weh’ und aber Weh’,
  nun weiß ich nicht wohin ich geh.
  Der Herr ſpricht über mich ein ſtrenges [\A2: grauſam] Urtheil aus,
  er jagt mich aus den Landen naus.
  Ich habe ſeinen Zorn durch meine That entzündet,
  und jeder ſchlägt mich todt, der mich ausfindet.
\end{movement}

\begin{movement}{}
  \voice[Cain]
  Ach, ach, die Größe meiner Sünden
  kan nimmermehr Vergebung finden
  vielleicht iſt auch die Reu zu ſpat.
  Wie ſeelig iſt ein gut Gewißen,
  das nicht darff ſolche [\A2: grobe] Fehler büßen,
  und das gar nichts verbrochen hat.
  [\A2: und ſowas nicht begangen hat.]
\end{movement}

\begin{movement}{}
  \voice[Eva]
  Wo kam denn Cain hin? Mir wird das Herze ſchwer.
  Was will das Blut und ſeine Flucht bedeuten?
  Was bringet Adam dort von weitem,
  und kommt ſo eilend her?

  \voice[Adam]
  Hier ſiehſt du nun, daß dein ſo gutes Hoffen
  dir leider nicht,
  nur aber mein Verdacht wohl eingetroffen.

  \voice[Eva]
  Was bringſt du da? Ach, was hat Cain angericht?

  \voice[Adam]
  Er hat den Abel todtgeſchlagen,
  und diß mit ſeinem Blut befleckte Kleid
  kan dir daran Gewißheit ſagen.

  \voice[Eva]
  O wie erhörtes Herzeleyd,
  o henkersgleiche That
  die Cains Mörderhand verübet hat,
  O höchſt betrübtes Angedenken,
  das eine Mutter kan biß an ihr Ende kränken.
\end{movement}

\begin{movement}{}
  \voice[Eva]
  Wen ſo ein Zufall nicht beweget,
  wes Herz ſich nicht durch Mitleid reget,
  hat keinen rechten Menſchenſinn.
  Es ſtirbt mein frömmſter Leibes Erbe,
  ein Wunder iſts, daß ich nicht ſterbe,
  und daß ich noch am Leben bin.
\end{movement}

\begin{movement}{}
  \voice[Adam]
  Ja freylich iſt wohl die Begebenheit
  auffs Höchſte zu bedauern.
  Wir haben Urſach, daß wir Lebenszeit
  deßwegen trauern.
  Gott ſelbſt hat nicht den Todt gemacht,
  daß Menſchen müßen ſterben,
  er hat nicht Luſt
  an der Lebendigen Verderben.
  Die Böſen haben ihn in dieſe Welt gebracht,
  weil ſie danach mit Wort und Werken ringen,
  und also kan er durch zu allen Menſchen dringen.
  Wir beyde ſind am erſten Schuld daran,
  denn hätten wir nicht unſern Fall gethan,
  ſo war ihm alle Macht benommen,
  und wäre nie zu uns auff Erden kommen.

  \voice[Eva]
  Ach leyder wird nun alles offenbahr,
  um unſret willen
  muß Abel uns zuerſt erfüllen.
  Es ſey nur allzu wahr,
  daß auch die Kinder müßen
  der Eldern Fehler [\A2: Sünden] büßen.
  Er war gerecht und ſtirbt,
  wie kommt es aber, daß die Böſen leben,
  die ſich doch wieder Gott erheben,
  und warum duldet er, daß Frömmigkeit verdirbt?
\end{movement}

\begin{movement}{}
  \voice[Chor]
  Der Fromme ſtirbt, der recht und richtig wandelt,
  der Böſe lebt, der wieder Gott mißhandelt,
  die Schuld bezahlt die Unſchuld, der Gerechte,
  ſtatt böſer Knechte.
\end{movement}

\begin{movement}{}
  \voice[Adam]
  Des Höchſten Wunder Wege
  ſind ſo geheimnißvoll,
  daß ſie der Menſch nicht unterſuchen ſoll,
  ſonſt führt ihn die Vernunfft auff lauter Irrthums Stege.
\end{movement}

\begin{movement}{}
  \voice[Chor]
  Man muß in Gottes Herz und Sinn
  ſein Herz und Sinn ergeben,
  ſo wird, was böſe ſcheint, Gewinn,
  und folgt dem Todt das Leben.
\end{movement}

\begin{movement}{}
  \voice[Adam]
  Dahero ſtelle man dem Herrn anheim
  das, was man nicht ergründen kan.
  Vielleicht giebt Abels Todt ſich als ein Vorbild an,
  es werde mit der Zeit auff Erden
  ein noch viel theuren Blut vergoßen werden,
  das beßer redt als Abels Blut,
  das gnug vor alle Sünden thut,
  das Gott verſöhnt, den wir zum Zorn erreget,
  und aller Menſchen Schuld bezahlt und träget.
\end{movement}

\begin{movement}{}
  \voice[Chor]
  Ein Lämmlein geht und trägt die Schuld
  der Welt und ihrer Kinder,
  es geht und büßet mit Gedult
  die Sünden aller Sünder,
  es geht dahin, wird matt und kran,k,
  ergiebt ſich auf die Würgebank
  entzieht ſich aller Freuden.
  es nimmt an ſich Schmerz, Hohn und Spott,
  Angſt, Wunden, Striemen, Creuz und Todt,
  und ſpricht: ich wills gern leiden.
\end{movement}

\begin{movement}{}
  \voice[Adam]
  Da wir uns nun darauff im Geiſte können freun,
  wie glücklich werden nicht die ſpäten Enkel ſeyn,
  die den Erlöſungstag erleben,
  den ihnen Gott
  zu aller Menſchen Heil wird geben.
\end{movement}

\begin{movement}{}
  \voice[Chor]
  Es ſchreyet Abels Blut um Rache
  und klagt den Bruder Mörder an.
  Ein jeder, der nicht wohlgethan,
  hat ſeinen großen Theil an einer böſen Sache,
  und wohl vielleicht den Fehler nie erkannt.
  Drum fehlt ihm auch die Reu, ſo ſich an [\A2: bey] Cain fand.
  Er ſchilt, und ſtrafft die Sündenwege,
  und ſetzet ſeinen Fuß doch ſelbſt auff ſolche Stege,
  er flucht auff Cains Übelthat,
  da er ihn ſelber doch bey ſich im Herzen hat.
\end{movement}
}

\eesScore

Wolfgang Hirschmann (2002). Metastasios Oratorientexte im Deutschland des 18. Jahrhunderts. Adaptionen und Transformationen. In: Laurenz Lütteken and Gerhard Splitt (eds.). Metastasio im Deutschland der Aufklärung. Bericht über das Symposium Potsdam 1999. Max Niemeyer Verlag, Tübingen. 217–246.

\end{document}
