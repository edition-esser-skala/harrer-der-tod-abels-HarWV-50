\documentclass[tocstyle=ref-genre]{ees}

\begin{document}

\eesTitlePage

\eesCriticalReport{
  –    & –       & –     & In \A1, lyrics are generally illegible.
                           Thus, they have been adopted from \A2. \\
       & –       & org   & In \A1, bass figures only appear in the following
                           movements and bars: 1.2 (110), 1.4 (28, 4th \quarterNote; and 32, 2nd \halfNote). All remaining bass figures have been added by the editor. \\
  1.1  & 23f     & vl 2, vla & bars empty in \A1 \\
  1.2  & –       & –     & Grace notes, trills, and slurs in the rhythms
                           \eighthNoteDotted-\thirtysecondNote–%
                           \thirtysecondNote and \sixteenthNoteDotted–%
                           \sixtyfourthnote–\sixtyfourthnote
                           have been tacitly emended.
                           Grace notes have been tacitly added in bars
                           16 (fag 2), 17 (ob grande 2), 20 (fl 1),
                           34 (fl, ob grande), 54 (fl), 58 (fag 1),
                           74 (fl 2, ob grande, vl 2), 75 (fl, vl),
                           131 (fag 1), 146 (ob grande 1), 173 (fl),
                           and 181 (fl 2, ob grande, vl 2, vla). \\
       & 5       & fl 2  & in \A1 unison with fl 1 \\
       & 7       & vl 1  & 2nd/3rd \eighthNote\ in \A1: g′4 \\
       & 7       & vl 2  & 2nd/3rd \eighthNote\ in \A1: e′4 \\
       & 11      & fl 2, vl 2 & 2nd \eighthNote\ in \A1: c″8 \\
       & 83      & fl 2  & 1st \eighthNote\ in \A1: g″8 \\
       & 119–121 & fl    & bars missing in \A1 \\
       & 123     & fl 1  & 1st \quarterNote\ in \A1: c′′′4 \\
       & 130     & ob grande 2 & 2nd \eighthNote\ in \A1: g′8 \\
       & 130     & fag 2 & 2nd \eighthNote\ in \A1: c8 \\
       & 171     & vla   & 1st \quarterNote\ in \A1: e′4 \\
}

\eesToc{
\begin{movement}{intro}
\end{movement}

\part{prima}

\begin{movement}{wunderbarer}
  \voice[Abel]
  O wunderbahrer Gott! an allen deinen Werken
  kan man die Allmacht merken.
  Ich will mein Leben lang [?]
  von deinem großen Nahmen ſingen
  und täglich ein Gelübde dir
  mit Ehrfurcht und viel Dank
  statt der Bezahlung bringen.
  Ihr Völker, preiſet ihn mit mir,
  laß’t ſeinen Ruhm erſchallen.
  Gott [\A2: Ihm] hat mein Opffer wohlgefallen.
  Ach, wer bin ich,
  mein [\A2: o] Schöpfer, daß du dich
  mein ſogar liebreich angenommen
  und ein elendes Menſchenkind
  ſo werthgeacht, daß es ganz unverdient
  zu ſolcher Gnade kommen.

  \voice[Cain]
  Warum iſt Abel ſo vergnügt?
  Was iſt ihm denn ſo gutes wiederfahren,
  daß Luſt und Wonne ſich in ſeinen Augen paaren?

  \voice[Abel]
  Ach theile doch mit mir,
  geliebter Cain, meine Freude,
  ach freuen wir uns beyde,
  weil ich mit Wahrheit ſagen kann,
  Gott ſah’ mein Opffer gnädig an.

  \voice[Cain]
  Und Abel unterſteht ſich das zu ſagen,
  du ſchmeichelſt dir zu viel.

  \voice[Abel]
  Wie ſoll ich etwan Zweiffel tragen?
  Umſonſt läßt Gott nicht was beſonders ſehn,
  wie dazumahl geſchehn.
  Denn als ich von den Erſtlingen der Heerde
  die feſten und die beſten nahm
  und Ihm, von dem doch alles kam,
  mit Danken und Gebeth ein Opffer brachte,
  ward ich etwas gewahr, das mich erſtaunend machte.
  Ich hatt’ es kaum auff den Altar gelegt,
  ward auch ſogleich [\A2: ſo ward auch gleich] des Himmels Kopff bewegt,
  ein Feuer fuhr herab gleich einem hellen Blitze,
  und durch deßalben Brand und Hitze
  ward alles in ſo kurzer Zeit verzehrt,
  als kaum ein ſchneller Blick von unſern Augen währt.

  \voice[Cain]
  Das iſt ja wunderſam! allein, es kan nicht fehlen,
  ſonſt würdest du mirs nicht erzehlen.
  Ich habe gleichfalls Gott von meiner Heerde Frucht
  ein Opffer zugedacht [\A2: ausgeſucht]
  und ihm wie du gebracht,
  doch ſowas ſeltens nicht erblicket.
  Und da es ſich ſo ſchicket,
  daß Eva kommt, ſo ſoll ſie es auch wißen,
  damit das ganze Hauß ein Feſt wird halten müßen.
  Komm, Eva, höre mich,
  ich habe dir was vorzubringen.
  Es redet Abel izt von lauter Wunderdingen,
  die Gott an ihm gethan.
  Nimms aber auch vor eine Wahrheit an.

  \voice[Eva]
  Du zweifelſt wohl vielleicht, ich hab’ es ſelbſt geſehen.

  \voice[Cain]
  Und was?

  \voice[Eva]
  Ein Feuer aus den Wolken gehen,
  dadurch das Opffer gleich verzehret ward.

  \voice[Cain]
  Iſt das gewiß?

  \voice[Eva]
  Was meine Augen ſchauen,
  drauff kanſt du ſicher bauen.

  \voice[Cain]
  O Wahrheit, die vor mich [?] nicht ſchrecklicher kann ſeyn.

  \voice[Eva]
  Ach Kinder, bildet euch darauff nichts ein,
  laß’t euch den Stolz nicht ſchmeichelnd nähren,
  laß’t euch die Mißgunſt nicht verführen.
  Denn jeder that wie er jeds wahl,
  den Dienst und Pflicht,
  den er Gott ſchuldig war. Vergeßet aus Gehorſam nicht,
  was Adam euch befahl;
  du Cain geh’ ins Feld, und Abel zu der Heerde,
  daß euer Thun verrichtet werde.
  Greifft alles Werk in Gottes Nahmen an,
  ſo iſt auch alles wohlgethan.
  Wenn wir mit Ihm die Hand zu Werke legen,
  ſo liebt er uns und ſchenkt uns ſeinen Seegen.

  \voice[Abel]
  Nichts angenehmers kan nur wohl geſchehn,
  als wenn man mich heißt zu der Heerde gehn.
  [\A2: als wenn ich ſoll zur Heerde gehn.]
  Sie iſt das einzige, an dem ich mich auch labe,
  ob ich dabey gleich Müh und Arbeit habe.
\end{movement}

\begin{movement}{ichbinein}
  \voice[Abel]
  Ich bin ein Hirte, der ſein Leben
  vor ſeine Schaaffe darzugeben
  aus Liebe kein Bedenken trägt.
  Sie hören meine Hirtenlieder,
  ich kenne ſie, und ſie auch wieder
  den Schäfer, der ſie zärtlich pflegt.
\end{movement}

\begin{movement}{wasfehlet}
  \voice[Eva]
  Was fehlet, Cain, dir?
  Du ſiehſt ja [\A2: Wie ſiehſt du] ſo zerſtöhrt in deinen Augen aus.
  Entdeck es mir
  und ſag es frey heraus.
  Du redeſt nichts und ſchlägſt die Augenlieder
  voll Unmuth und Verdruß zur Erde nieder.

  \voice[Cain]
  Es geht mir nicht darnach, daß ich kan fröhlich ſeyn.

  \voice[Eva]
  Wie [\A2: Ey] ſolle dich des Bruders Ehre nicht erfreun!

  \voice[Cain]
  Die eben iſts, die mich empfindlich kränket.

  \voice[Eva]
  Iſts möglich, daß dein Herz ſo übel denket,
  ſo läßt du dich denn des verdrießen,
  was wir vor Gottes Gnade ſchützen müßen?
  Dies iſt des Schöpffers Art, der iſt ein Schadenfroh.
  Mein Cain, haader doch nicht ſo.
  Wir müßen unſern Nächſten ja was Gutes gönnen,
  weil wir dergleichen allewohl auch hoffen können.
  Reiß dieſes Unkraut aus, eh es auf Wurzel faßet.
  Wenn Cain ſeinen Bruder haßet,
  ſo wird ſich täglich etwas andres finden,
  worauff er ſeinen Haß kann gründen.
  Mein Sohn, ach thu mir doch ſolch Herzeleid nicht an,
  daraus noch mit der Zeit was ärgers kommen kan.
\end{movement}

\begin{movement}{einwasser}
  \voice[Eva]
  Ein Waßer, das aus trüben Quellen
  pflegt ſeine Fluthen herzuſtellen,
  laufft niemahls klar ins Meer hinein.
  Wie will es alſo hier auff Erden
  nah mit euch beyden Brüdern werden,
  wenn dieſes ſoll der Anfang seyn?
\end{movement}

\begin{movement}{sosollder}
  \voice[Cain]
  So ſoll der junge Bruder denn
  vielmehr als ich der Ältre ſeyn,
  und ich dabey ganz unempfindlich bleiben,
  daß heißet die Geduld auffs allerhöchſte treiben.
  Er hat die Ehre nur allein,
  ich aber nichts als Schimpff und Schande,
  und niemand klaget mich in ſo betrübtem Stande.
  Ja fang ichs noch ſo liſtig an
  und will ihn nur bedrücken,
  muß ſich doch etwas ſchicken,
  daß ich auch an ihn kommen kan,
  dadurch wird er von Zeit zu Zeiten größer,
  allein mit mir wirds niemahls beßer.
  Es wächſt ſein Übermuth, in den ich mit Bedruß,
  der kaum zu dulden ist, doch alles leiden muß.
\end{movement}

% \begin{movement}{}
%   \voice[]
% \end{movement}

% \begin{movement}{}
%   \voice[]
% \end{movement}

% \begin{movement}{}
%   \voice[]
% \end{movement}

% \begin{movement}{}
%   \voice[]
% \end{movement}
}

\eesScore

\end{document}
